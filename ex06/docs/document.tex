\documentclass[]{scrartcl}
\usepackage{Preamble}

\setcounter{section}{6}
\newcommand{\exercise}{Exercise \thesection}
\newcommand{\duedate}{2021-01-18, 23:59}

\begin{document}
\section*{\exercise}

To compile: unzip our uploaded code, and run \verb|make| inside \verb|code/|.
The slurm scripts are stored inside \verb|code/slurm/|.

To debug: run the debug outputs (\verb|*.dbg|) and attach gdb to respective pids

\subsection{Heat Relaxation II --- Parallel Implementation}\label{ssec:impl}
\subsection{Heat Relaxation II --- Experiments}

The implementation in \autoref{ssec:impl} resulted in the values below (\autoref{tabl:heat_t}, \autoref{tabl:heat_s}, and \autoref{tabl:heat_e}).
We choose the by-slot-Mapping (default of mpirun) as to reduce the number of hops between nodes.
This Mapping starts filling a node's possible slots with ranks until full and then continues with another node.
This results in rank 3 and 4 and ranks 7 and 8 communication between nodes (for 9 or more ranks).

\begin{table}[ht]
  \caption{Time [$\mu$s] / iteration}\label{tabl:heat_t}
  \begin{tabular}{lrrrrrrr}
\toprule
{} &  NP = 01 &  NP = 02 &  NP = 04 &  NP = 06 &  NP = 08 &  NP = 10 &  NP = 12 \\
Grid size &          &          &          &          &          &          &          \\
\midrule
128x128   &       68 &       39 &       44 &      526 &     2373 &      693 &      743 \\
512x512   &     3363 &      851 &      757 &     3007 &      932 &     1417 &     3294 \\
1024x1024 &    19474 &     9524 &     5684 &     6003 &     2722 &     3068 &     2573 \\
2048x2048 &   238361 &   134284 &    30991 &    22886 &    16389 &    13306 &    13621 \\
4096x4096 &   855031 &   551723 &   365791 &   130806 &    78609 &    68622 &    61378 \\
\bottomrule
\end{tabular}

\end{table}

\begin{table}[ht]
  \caption{Speedup}\label{tabl:heat_s}
  \begin{tabular}{lrrrrrrr}
\toprule
{} &  NP = 01 &  NP = 02 &  NP = 04 &  NP = 06 &  NP = 08 &  NP = 10 &  NP = 12 \\
Grid size &          &          &          &          &          &          &          \\
\midrule
128x128   &   1.0000 &   1.7289 &   1.5388 &   0.1299 &   0.0288 &   0.0987 &   0.0920 \\
512x512   &   1.0000 &   3.9487 &   4.4397 &   1.1185 &   3.6086 &   2.3725 &   1.0209 \\
1024x1024 &   1.0000 &   2.0447 &   3.4260 &   3.2438 &   7.1544 &   6.3462 &   7.5672 \\
2048x2048 &   1.0000 &   1.7750 &   7.6911 &  10.4150 &  14.5433 &  17.9135 &  17.4986 \\
4096x4096 &   1.0000 &   1.5497 &   2.3375 &   6.5366 &  10.8769 &  12.4600 &  13.9305 \\
\bottomrule
\end{tabular}

\end{table}

\begin{table}[ht]
  \caption{Efficiency}\label{tabl:heat_e}
  \begin{tabular}{lrrrrrrr}
\toprule
{} &  NP = 01 &  NP = 02 &  NP = 04 &  NP = 06 &  NP = 08 &  NP = 10 &  NP = 12 \\
Grid size &          &          &          &          &          &          &          \\
\midrule
128x128   &   1.0000 &   0.8645 &   0.3847 &   0.0217 &   0.0036 &   0.0099 &   0.0077 \\
512x512   &   1.0000 &   1.9743 &   1.1099 &   0.1864 &   0.4511 &   0.2373 &   0.0851 \\
1024x1024 &   1.0000 &   1.0224 &   0.8565 &   0.5406 &   0.8943 &   0.6346 &   0.6306 \\
2048x2048 &   1.0000 &   0.8875 &   1.9228 &   1.7358 &   1.8179 &   1.7913 &   1.4582 \\
4096x4096 &   1.0000 &   0.7749 &   0.5844 &   1.0894 &   1.3596 &   1.2460 &   1.1609 \\
\bottomrule
\end{tabular}

\end{table}

\begin{itemize}
  \item A speedup is observed, that correlates to the number of jobs (i.e.\ for 10 jobs we reach a speedup o approx 10, for a sufficiently large problem size)
  \item For problem sizes too small, performance drops, due to the communication overhead dominating
  \item Additionally super linear speedups were observed, probably due to better cache utilization.
\end{itemize}

\subsection{Heat Relaxation II --- Tracing}

\end{document}
