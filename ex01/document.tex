\documentclass[]{scrartcl}
\usepackage{../template/Preamble}

\setcounter{section}{1}
\newcommand{\exercise}{Exercise \thesection}
\newcommand{\duedate}{2020-11-16, 23:59}

\begin{document}
\section*{\exercise}
\subsection{Reading}
\subsection{Moore's Law}
\subsubsection{}

\subsection{Amdahl's Law}
\subsubsection{}
\begin{itemize}
    \item new CPU 10 times faster
    \item old CPU spent 40\% of execution time on calculations
    \item remaining time was for IO
\end{itemize}
\begin{align}
    S &:= 60\%\\
    P &:= 40\%\\
    N &:= 10\\\nonumber\\
    Speedup &= \frac{1}{.6+\frac{.4}{10}} = 1.563
\end{align}

$\Rightarrow$ We would expect a 56\% performance improvement from the new CPU\@.

\subsubsection{}
\begin{itemize}
    \item 20\% of compute time is used for squaere roots
    \item possibilities:
        \begin{itemize}
            \item improve floating point square root calculations by factor of 10
            \item improve all fp operations by 1.6
        \end{itemize}
    \item 50\% of operation is spent on FP
\end{itemize}
\begin{align}
    S_1 &= \frac{1}{(1-(0.5\cdot 0.2))+\frac{0.5\cdot0.2}{10}}\\
        &= 1.099\\\nonumber\\
    S_2 &= \frac{1}{(1-0.5)+\frac{0.5}{1.6}}\\
        &= 1.231
\end{align}
$\Rightarrow$ By accelerating all FP operations by a facctor of 1.6 a speedup of 23\% can be observed and therefore is the optimal solution (in contrast to only 9.9\% when only speeding up FPSQRT).

\subsubsection{}
\begin{align}
    100 &= \frac{1}{(1-P)+\frac{P}{128}}\\
    \Leftrightarrow P&= 0.9978\\
    \Rightarrow S &\leq 0.22\%
\end{align}
\end{document}
