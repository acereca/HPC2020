\documentclass[]{scrartcl}
\usepackage{../template/Preamble}

\setcounter{section}{1}
\newcommand{\exercise}{Exercise \thesection}
\newcommand{\duedate}{2020-11-16, 23:59}

\begin{document}
\section*{\exercise}
\subsection{Reading}
\subsection{Moore's Law}
\subsubsection{}
Apply Moore's Law to currently fastest Supercomputer to extrapolate time until exa
scale performance is achieved.
\begin{enumerate}
	\item Consider derived law stating that computing power doubles every 18 months:
	\begin{equation}
		P_{\textrm{compute}}(t) = N_0 2^{\frac{1}{18} t}
	\end{equation}
	where $ N_0 $ is the computing power at time 0 and $ t $ the time in months.
	\item Set $ P_{\textrm{compute}} $ to \SI{1e18}{flop\per\second}.
	\item Set $ N_0 $ to current max performance of \SI{415530e12}{flop\per\second}
	(\emph{Supercomputer Fugaku})\footnote{https://top500.org/lists/top500/2020/06/}.
\item Solving for $ t $ yields a time of $ \approx23 $ months (see figure~\ref{fig:Moore}).
\end{enumerate}
$ \Rightarrow $ extrapolating from current performance using a derived Moore's law
Exa scale computing power will be achieved in approximately 23 months or almost two years.
\begin{figure}[htpb]
	\centering
	\includegraphics[width=0.8\linewidth]{./plots/Moore}
	\caption{Extrapolating time until exa flop from derived Moore's law}%
	\label{fig:Moore}
\end{figure}
\subsection{}
Determine time until exa scale from growth rate from TOP500 list
\begin{enumerate}
	\item Use data from 2007 and 2011
	\item Linear fit (on log scale) yields that exa flop performance should
		have been reached around 2018 (see figure~\ref{fig:GrowthRate})
\end{enumerate}
\begin{figure}[htpb]
	\centering
	\includegraphics[width=0.8\linewidth]{./plots/GrowthRate.pdf}
	\caption{Extrapolating time until exa flop from TOP500 data using linear fit}%
	\label{fig:GrowthRate}
\end{figure}

\newpage
\subsection{Amdahl's Law}
\subsubsection{}
\begin{itemize}
    \item new CPU 10 times faster
    \item old CPU spent 40\% of execution time on calculations
    \item remaining time was for IO
\end{itemize}
\begin{align}
    S &:= 60\%\\
    P &:= 40\%\\
    N &:= 10\\\nonumber\\
    Speedup &= \frac{1}{.6+\frac{.4}{10}} = 1.563
\end{align}

$\Rightarrow$ We would expect a 56\% performance improvement from the new CPU\@.

\subsubsection{}
\begin{itemize}
    \item 20\% of compute time is used for squaere roots
    \item possibilities:
        \begin{itemize}
            \item improve floating point square root calculations by factor of 10
            \item improve all fp operations by 1.6
        \end{itemize}
    \item 50\% of operation is spent on FP
\end{itemize}
\begin{align}
    S_1 &= \frac{1}{(1-(0.5\cdot 0.2))+\frac{0.5\cdot0.2}{10}}\\
        &= 1.099\\\nonumber\\
    S_2 &= \frac{1}{(1-0.5)+\frac{0.5}{1.6}}\\
        &= 1.231
\end{align}
$\Rightarrow$ By accelerating all FP operations by a facctor of 1.6 a speedup of 23\% can be observed and therefore is the optimal solution (in contrast to only 9.9\% when only speeding up FPSQRT).

\subsubsection{}
\begin{align}
    100 &= \frac{1}{(1-P)+\frac{P}{128}}\\
    \Leftrightarrow P&= 0.9978\\
    \Rightarrow S &\leq 0.22\%
\end{align}
\end{document}
