\documentclass[]{scrartcl}
\usepackage{Preamble}
\usepackage{tikz}
\usepackage{pstricks}

\setcounter{section}{8}
\newcommand{\exercise}{Exercise \thesection}
\newcommand{\duedate}{2021-02-01, 23:59}

\begin{document}
\section*{\exercise}

\subsection{Reading}
\subsubsection{On the Memory Access Patterns of Supercomputer Applications: Benchmark Selection and Its Implications}
This paper tries to compare a set of Benchmarks to codes, run on the Sandia National Laboratories, consuming a significant amount of their computation time.
Murphy and Kogge try to quantify the three key characteristics of applications (temporal and spacial locality, as well as data intensiveness) in terms of memory access.
This is done in an effort to be able to select benchmarks for supercomputing systems, when taking real world applications into consideration.

The authors ran an extensive set of both FP and Integer, as well as already well established Benchmarks (like SPEC and STREAM), to have a wide spectrum of resulting localities and intensiveness to compare to.
Additionally they defined their key characteristics in a quantitatively defined way and then compared Benchmarks and real-world applications based on those.
They concluded, that real world integer applications are ``uniformly harder on the memory system than the SPEC integer suit'' and that ``architects should focus on codes with significantly larger data set sizes'' when benchmarking.

Murphy and Kogge's observations are reasonable and clearly shown, we therefore accept this paper.


\subsubsection{On the Effects of Memory Latency and Bandwidth on Supercomputer Application Performance}

\subsection{n-Body Problem --- Implementation}
\subsection{n-Body Problem – Experiments}
\end{document}
